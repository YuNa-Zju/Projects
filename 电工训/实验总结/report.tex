\documentclass[UTF8]{ctexart}
\usepackage{geometry}
\geometry{top=0.8in,bottom=1in,left=2cm,right=2cm}
\linespread{1.2}
\title{电子工程训练课程总结报告}
\author{何铭源}
\date{\today}
\begin{document}
\maketitle
\section{引言}
为期一学期的电子工程训练课程,是一次理论与实践紧密结合的宝贵学习经历。
这门课程内容丰富、循序渐进,涵盖了从最基础的元器件焊接到复杂的机电一体化系统调试的全过程。
通过亲自动手操作,我不仅将在课堂上学到的理论知识应用于实践,更深刻地理解了电子系统的设计、构建、调试与优化的完整流程。
整个课程的学习大致可分为三个核心部分:工训基础、软硬件联调以及机电一体化系统搭建。本次总结将围绕这三个部分,详细回顾我的学习过程、技能收获与心得体会。

\section{工训基础}
工训基础部分是整个课程的起点,其核心目标是让我们掌握电子制作与测量的基本功。这一阶段的学习,为后续更复杂的项目打下了坚实的基础。

\subsection{焊接技术的掌握与实践}
在课程初期,我系统学习了手工焊接的基本要领,从如何正确握持电烙铁、送锡,到判断焊点的优劣。在不断的练习之下,我逐渐克服了手抖、虚焊、连焊等初学者常见的问题。
更具挑战性的是贴片元器件的焊接。与传统的直插式元器件相比,贴片元件体积小、引脚间距窄,对焊接精度和技巧要求更高。
在“贴片流水灯”的制作中,我首次接触并实践了贴片元件的焊接方法。起初,面对米粒大小的电阻和IC芯片,我感到无从下手。
但在老师的指导和反复尝试下,我成功的实现了贴片电阻的焊接。这个过程极大地锻炼了我的耐心和细心。

\subsection{常用仪器的精通与应用}
在这一部分,我重点学习了万用表、直流稳压电源、信号发生器和数字示波器的使用。

万用表:在实验中,我使用万用表测量了不同阻值的电阻和不同容值的电容,并通过与标称值对比计算误差。

电源与示波器:这是本阶段学习的重点与难点。我练习了如何精确设定电源的输出电压与限流值,并验证了电源的串联模式以获得倍压输出。
而示波器的使用对我来说是个全新的只是。
我从最基础的校准信号观察开始,学会了使用Autoset功能设置示波器,并利用Cursor模式精确测量信号的峰值、周期和频率占空比等关键信息。

\subsection{从焊接到调试}
“幸运转盘”项目是我综合运用本阶段所学知识的第一次“实战”。
在成功焊接电路板后,我利用示波器对电路的关键节点进行了波形观测和分析。
例如,我测量了U2芯片的输出信号,其周期约为170ms,脉冲宽度为17ms,由此计算出占空比约为10\%。
通过示波器屏幕上跳动的波形,我第一次“看”到了芯片内部的工作状态,将抽象的电路图与现实的电信号活动联系了起来。

\section{软硬件联调}
在掌握了基础的硬件技能后,课程进入了更具挑战性的软硬件联调阶段。以“智能插座”项目为载体,我学习了如何将硬件电路与软件编程相结合,打造一个具有初步“智能”的电子系统。

\subsection{智能插座的硬件搭建与调试}
智能插座的电路板比之前的练习项目复杂得多,包含了供电模块、继电器控制模块、传感器模块和微控制器接口等。焊接过程本身就是一次综合考验。完成后,我又根据PPT进行了一系列的分模块调试,其中包括:
供电电路测试、
控制电路测试、
芯片供电检测等模块。
这个过程让我明白,一个复杂的系统并非一蹴而就,而是需要逐个模块击破,确保每部分都无误后才能进行系统联调。

\subsection{数据标定与软件校准}
这是本阶段最能体现“软硬件结合”思想的部分。硬件传感器(如温度传感器)的输出值不太准确,无法直接使用。我们需要通过软件算法进行标定和校准。

温度标定:通过调节可变电阻,我将程序显示的温度值从28.6$\textcelsius$校准到了与实际室温接近的31.0$\textcelsius$。

电流标定:这是一个更具挑战性的任务。我发现串口直接读取的电流值与电子负载显示的实际值偏差巨大(如读取值457mA对应实际值78.3mA)。
为了解决这个问题,我采集了多组数据,利用线性拟合,计算出串口读数值($x$)与实际电流值($y$)之间的线性关系式:$y=0.1737x−1.021$。随后,我将这个公式写入Arduino代码中,对传感器原始数据进行实时计算和修正。
经过校准后,测量误差显著降低,基本在0.5\%以内,这让我深刻体会到了软硬件的互相配合。

\subsection{系统联调与功能实现}
在完成了软硬件的各自调试和数据标定后,最终的目标是实现一个功能完善的智能插座,并通过手机APP进行控制。我通过老师提供的APP控制插座的通断、读取实时电流、电压、功率和温度数据,并测试了定时/延时开关、告警及自动断电保护等高级功能。

\section{机电一体化}
课程的最后部分,是集机械、电子、控制与编程于一体的“机电一体化”项目。

\subsection{智能小车的搭建}
小车的搭建从底盘和电机组装开始,这让我对机械结构有了一定的了解。

通过在小车上安装树莓派,语音模块和红外传感器,超声传感器,小车具有了一定的避障能力和循迹能力。

\subsection{机械臂与视觉的融合}
这是整个电工训课程中最具挑战性、也最令我兴奋的项目。我在小车上加装了一个机械爪和一个摄像头,目标是让小车能够自主寻找并抓取特定颜色的木块。

通过这个项目,我体会到了“舵机”这一关键机械部件的功能,也体会了摄像头与机械爪的配合,通过摄像头计算木块的位置和大小信息,引导小车去抓握木块。

\section{总结与展望}
回顾整个电子工程训练课程,我收获的远不止几块亲手制作的电路板和一台会动的小车。更重要的是,我收获了实际动手解决工程问题的经历和调试问题,解决问题的能力。

从基础焊接开始,我学会了严谨与精细;通过仪器测量,我学会了观察与分析;在软硬件联调中,我理解了协同与优化;在最终的机电一体化项目中,我领略了集成的魅力。课程的教学安排非常得当,由浅入深,环环相扣。

电工训课程为我打开了一扇通往广阔电子世界的大门。它让我真切地感受到,我们接下来的学习方向和目标。
\end{document}