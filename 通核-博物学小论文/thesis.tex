\documentclass{source/Paper}
\usepackage{chngcntr}
\counterwithin{footnote}{page} % 每页重置脚注计数器
\usepackage{pifont}
\renewcommand{\thefootnote}{\ifcase\value{footnote}\or\ding{172}\or\ding{173}\or\ding{174}\or\ding{175}\or\ding{176}\or\ding{177}\or\ding{178}\or\ding{179}\or\ding{180}\or\ding{181}\else\@ctrerr\fi}
\articletitle{从海错图看清初民间博物学建构}
\name{何铭源}
\stuid{3240104481}
\Abstract{}
\Keyword{清初; 民间博物学; 海错图; 物种分类; 知识建构}
\begin{document}
\makeheader
\section{引言}
清代康熙年间,一部名为《海错图》的海洋生物图谱横空出世,
成为中国古代博物学领域的一颗璀璨明珠
。此书由民间博物学爱好者聂璜历时数十年,于康熙三十七年(1698年)左右完成,
共计四册,收录了三百七十余种海洋生物及相关物事。\footnote{}所谓“海错”,取自古语,
意指海洋中种类繁多、错综复杂的生物与物产。聂璜,字存庵,浙江钱塘(今杭州)人,
是一位活跃于明末清初的画家及生物学爱好者。他并非科举出身的士大夫,却凭借对海洋世界的浓厚兴趣与不懈探索,
游历中国东南沿海各地,亲身观察、多方询访、考证典籍,最终将所见所闻汇集成这部图文并茂的巨著。

《海错图》的流传经历颇具传奇色彩。这部源自民间的创作,
在雍正年间由太监苏培盛带入宫中,深得清朝历代帝王,
尤其是乾隆皇帝的喜爱与重视。乾隆帝不仅命人重新装裱,
钤上“乾隆御览之宝”、“重华宫鉴藏宝”等御玺,还将其著录于《石渠宝笈续编》,
彰显了其在清宫收藏中的重要地位。
这种从民间创作到皇家珍藏的历程,凸显了《海错图》独特的价值,
它不仅是聂璜个人求知热情的结晶,也反映了清代上层社会对自然知识与精细图谱的兴趣与吸纳,成为连接民间智慧与宫廷文化的桥梁。

\section{聂璜的探索之路}
\subsection{经验与博学的融合}
聂璜的《海错图》并非书斋中的空想之作,而是其数十年如一日辛勤付出的成果。
他“历经几十年,访遍全国各地江海湖泊,考察积累,绘制而成”,其足迹遍及河北、天津、浙江、福建等沿海地区\footnote{}。
这种深入实地的考察构成了他知识来源的基石,他强调“亲眼所见、亲耳所闻”\footnote{},力求描绘的真实性。

除了亲身观察,聂璜非常重视向当地居民,尤其是渔民和海客学习。
他“每看到一种新的海洋生物,就画下来,并查阅相关的资料,或去请教当地的渔民”\footnote{}。
这种对地方性知识和口头传统的尊重与采纳,
使得《海错图》不仅记录了生物形态,还融入了丰富的民俗信息和实践经验。

同时,聂璜并非罔顾前人成果。
他广泛涉猎古代典籍,如《山海经》、《广东新语》等,
并将自己的观察与文献记载进行比对考证。值得注意的是,
他还接触并引用了一些受到西方知识影响的汉文著作,如《西洋怪鱼图》和艾儒略的《西方答问》\footnote{}。
尽管他认为这些西学文献“但纪者皆外洋国族,所图者皆海洋怪鱼,于江浙闽广海滨所产无与也” \footnote{},即其内容多为域外之物,
与中国沿海生物不尽相符,但这种关注本身已显示出其开放的学术视野和试图整合不同知识来源的努力。
聂璜的治学方法,实质上是一种综合性的知识获取策略,体现了在没有严格学科划分的时代,
一位博物学爱好者如何融合个人观察、地方智慧与文献考据,构建其独特的知识体系。
这种方法可被视为一种广博探索。

\subsection{想象与传闻}
在十七、十八世纪,科学观察手段尚不发达,
人类对自然界的认知,特别是对遥远或隐秘领域的认知,
很大程度上依赖于间接信息。因此,传闻和想象在当时的知识体系中扮演着不可或缺的角色。
聂璜在《海错图》中,不仅记录“所见”,也大量采纳“所闻”\footnote{}。
对于一些难以证实或超出其经验范围的生物,他会注明信息来源,甚至坦承自己的困惑,
留下“以俟后有博识者辨之”的字句。
这种做法,并非单纯的猎奇或迷信,而是在特定历史条件下,一种试图全面记录、理解复杂世界的努力。
它反映了一种求知的审慎态度:在无法亲证的情况下,忠实记录传闻,并期待后人能够进一步辨析。
这种坦诚面对知识局限性的态度,反而增强了其作品的史料价值,使我们得以窥见当时知识建构的真实过程。

\section{案例分析}
《海错图》中的生物描绘,并非整齐划一地遵循现代生物学的写实标准,而是呈现出一个从精确观察到奇幻想象的广阔光谱。
本节将选取若干典型案例,分析聂璜如何在观察、传闻与想象之间进行取舍与融合。

\end{document}