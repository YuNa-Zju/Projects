1911年5月,清政府颁布``铁路国有''政策,
将已归商办的川汉、粤汉铁路收归国有,并以路权为抵押向英、法、德、美四国银行团借款。
这一政策表面上以``统一规划''为名,实则通过牺牲地方利益换取列强支持,激起了四川、湖南、湖北、广东四省民众的强烈反对,
尤其是四川保路运动迅速演变为全民抗议,参与者涵盖士绅、商人、农民等阶层,形成``租股''形式的经济动员网络。
\autocite{zhihu}
与此同时,湖北革命党人长期在新军中渗透,以文学社和共进会为核心的组织已发展数千名成员,占湖北新军总人数的30\%。
1911年10月10日,武昌新军因清廷调兵入川镇压保路运动导致防务空虚,仓促发动起义,仅凭20颗子弹攻占楚望台军械库,
最终促成亚洲第一个民主共和国——中华民国的诞生。