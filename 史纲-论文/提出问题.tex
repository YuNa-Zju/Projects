\subsection{问题提出的背景}
保路运动与武昌起义的时空关联性历来是近代史研究的核心议题。既有研究多集中于保路运动的民族主义性质或武昌起义的军事偶然性,
却较少系统分析二者间的联动机制。
具体而言,了解保路运动和武昌起义的关联性需解答以下问题:

\begin{enumerate}
    \item \textbf{经济冲突的政治化}:铁路“租股”涉及农民利益,清政府以“国有化”名义侵吞民间资本,其政策执行中的官僚腐败如何激化社会矛盾?

    \item \textbf{军事调动的战略真空}:清廷调派湖北新军1.6万入川镇压保路运动,导致武汉防务仅余5000兵力,革命党如何利用这一契机调整起义计划?

    \item \textbf{革命组织的动员效率}:文学社与共进会如何通过铁路网络与电报系统协调行动?武昌起义的仓促发动是否暴露了清政府对信息控制的失效?
\end{enumerate}
\subsection{解决的思路}
本研究将以\textbf{“经济权益—军事调动—信息网络”}为分析框架,结合清廷档案、地方志与革命党回忆录,
揭示保路运动不仅是地方性经济抗议,更是通过军事、信息与组织网络的联动,成为终结帝制的系统性危机的关键节点。