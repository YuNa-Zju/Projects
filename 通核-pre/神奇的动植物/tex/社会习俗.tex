\subsection{``蚰蜒卦''与占卜}

明代《五杂俎》载,某些地区通过观察蚰蜒爬行轨迹占卜(称``蚰蜒卦''),但其法已失传,仅存``虫迹如谶''的模糊记载。

关于《五杂俎》:《五杂俎》是明代谢肇淛创作的一部著名的随笔札记。
内容包括读书心得和事理的分析,也记载政局时事和风土人情,涉及社会和人的各个方面。

\subsection{入耳禁忌}
入耳禁忌是指古人认为蚰蜒进入耳朵后会导致出现头痛、疯癫等症状

明·陆容《菽园杂记》:北方有虫名蚰蜒,状类蜈蚣而细,好入人耳。
闻之同僚张大器云:人有蚰蜒入耳不能出,初无所苦,久之觉脑痛。疑其入脑,甚苦之,而莫能为计也。
一日将午饭,枕案而睡,适有鸡肉一盘在旁,梦中忽喷嚏,觉有物出鼻中,视之,乃蚰蜒在鸡肉上,自此脑痛不复作矣。
又同僚苏文简在山海关时,蚰蜒入其仆耳。
文简知鸡能引出,急炒鸡置其耳旁,少顷,觉有声鍧然。乃此虫跃出也。

关于《菽园杂记》:此书对明代朝野故实叙述颇详,而且较少抄袭旧文,论史事、叙掌故、谈韵书、说文字,皆大多为自己的见解,
被他同时代的王鏊称为明朝记事书第一。