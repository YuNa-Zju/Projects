\subsection{阴湿污秽之物}
蚰蜒常出没于阴暗潮湿的墙角、朽木或废墟,其存在被视为环境脏乱、家宅不宁的征兆。
\begin{enumerate}
    \item 如清代《燕京岁时记》 载:``蚰蜒上墙,家宅必败'',民间建房时甚至刻意用石灰涂抹墙缝以驱虫辟秽。

    《燕京岁时记》是清代一部详细记录北京(旧称``燕京'')岁时节令、民俗风物的笔记体著作,
    由满族文人富察敦崇(字礼臣)于光绪三十二年(1906年)编纂成书。
    全书以农历时序为纲,按月分述北京地区的节日习俗、物产饮食、民间信仰及市井生活,是研究晚清北京民俗文化的重要文献。

    \item 《九思·哀岁》巷有兮蚰蜒,邑多兮螳螂 。(最早的出处)

    这句话的翻译是:``街巷之中潮虫遍地,城镇之上充斥螳螂。 ''

    \item 《榆社县志》二月一日,各家用灰围住房子,煮芥菜来驱赶蚰蜒。东人呼皆今蚰蜒,喜入耳者也。
\end{enumerate}

\subsection{讽刺之用}

\begin{enumerate}
    \item 《聊斋志异》卷十一 第15《蚰蜒》:``学使朱矞三家门限下有蚰蜒,长数尺。每遇风雨即出,盘旋地上如白练然。按蚰蜒形若蜈蚣,昼不能见,夜则出,闻腥辄集。或云:``蜈蚣无目而多贪也''

    文中``学使朱矞三''即朱雯,字三,石门(今浙江桐乡市)人。康熙三年(1664)二甲第三十名进士,历官孝感知县、江宁同知。康熙三十年(1691)以山东按察司副使提督全省学政,迁济东道。蒲松龄对朱雯颇有微词,在此文中指名道姓,将其比作蚰蜒(形类蜈蚣),讽刺他``无目多贪''``闻腥辄集'',可谓辛辣至极。

    \item 《法华义疏\footnote{佛教词典}》记载:``重嗔如蚖蛇蝮蝎,轻嗔如蜈蚣蚰蜒。
\end{enumerate}

\subsection{经济生活的双面符号}

\begin{enumerate}
    \item 吉兆:北方部分地区称蚰蜒为``钱串子'',认为其出现预示财富积累(因形似串起的铜钱),山西某些地区甚至忌杀蚰蜒,以免``断了财路''。
    \item 凶兆:江南民间则认为蚰蜒是``破财虫'',若爬过钱柜或账本,需立即焚香驱赶,以防家财流失。
\end{enumerate}

\subsection{纠缠与琐碎之事的隐喻}

红楼梦中有如下场景:《红楼梦》第三九回:那焙茗去后,宝玉左等也不来,右等也不来,急的热地里的蚰蜒似的。

\subsection{指曲折蜿蜒}

\begin{enumerate}
    \item 造字法
    \item 老人``蚰蜒路''的说法
\end{enumerate}