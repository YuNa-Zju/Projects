\section{商品拜物教的起源}
马克思提出的“商品拜物教”概念,植根于19世纪资本主义生产方式的独特历史土壤及其对古典政治经济学的深刻批判。
\subsection{19世纪资本主义的社会经济图景}
19世纪是资本主义生产方式在西欧取得支配地位并引发社会结构深刻变革的时代。
工业革命是这一时期最显著的特征,它彻底改变了物质生产的面貌\autocite{jinji}。
这种转变的核心在于,生产的目的不再仅仅是为了满足生产者自身的需求,而是为了在市场上进行交换,即商品的普遍化生产\autocite{jinji}。

与生产方式变革相伴的是社会阶级结构的重塑。
工业革命或资本主义生产方式催生并巩固了两大对立的社会阶级:
掌握生产资料的资产阶级和除了自身劳动力外一无所有、必须出卖劳动力以换取工资的无产阶级。
这种阶级关系是资本主义生产关系的核心,它规定了社会财富的生产与分配方式。
这一时期的经济局面,尤其是在英国,为资本主义的兴盛创造了极为有利的条件,技术改良迅速扩大了劳动生产率,
并带来了大量增加的无产阶级队伍和广阔的投资范围。

市场经济体制的确立是19世纪资本主义的另一关键特征。
自由主义经济思想盛行,市场在资源配置中发挥主导作用,商品生产和交换成为社会经济活动的主要形式。
在这种背景下,人与人之间的社会关系日益被商品交换关系所中介。
然而,这种新型的社会经济形态也带来了新的社会矛盾和认识上的迷雾。
大工厂制度使得商品似乎脱离了具体的生产者而独立存在,市场交换的匿名性使得劳动者与最终消费者之间的联系变得间接和模糊。
传统社会中相对透明的生产关系,在资本主义条件下被复杂的商品交易所掩盖。
正是这种生产社会化与生产资料私人占有之间的矛盾,以及由此产生的社会关系物化现象,构成了马克思提出商品拜物教概念的直接历史背景。
新的社会现实要求一种新的理论工具来揭示其内在的运行逻辑和被遮蔽的本质。

这种19世纪工业资本主义带来的生产与社会生活的剧变,使得支撑经济活动的实际人类劳动和社会互动变得日益模糊。
大规模生产、匿名市场以及生产者与消费者的分离,使得人们难以直接感知经济活动的社会本质。
古典政治经济学虽然分析了市场机制,但未能充分揭示这种社会关系的神秘化。
因此,马克思需要“商品拜物教”这一概念来诊断这种新的社会“盲点”或幻象,即人际关系表现为物际关系。
这不仅仅是对价值的观察,更是对资本主义如何重构人类感知和社会理解的深刻批判。


\subsection{马克思对政治经济学的批判与概念的引入}
马克思的商品拜物教理论是在其对古典政治经济学进行深入批判的过程中形成的。
他并非凭空创造此概念,而是借鉴并转化了既有的思想资源。
马克思认为,拜物教是“感性欲望的宗教”,幻想使得拜物者相信无生命物体能满足其欲望。

马克思在1867年出版的《资本论》第一卷第一章“商品”的第四节“商品的拜物教性质及其秘密”中,正式提出了商品拜物教的概念。
他运用这一概念来阐释资本主义社会中劳动力的社会组织如何体现在商品的买卖过程中。
马克思指出,一件商品“乍看起来是一个极普通、极平凡的东西。
但是,对商品的分析表明,它却是一个充满形而上学的微妙和神学的怪诞的东西” \autocite{marx}。
他借用“拜物教”来批判当时的政治经济学家,认为他们将资本主义的经济范畴视为永恒的自然属性,从而掩盖了其历史性和社会性。

马克思之所以选择“拜物教”这一源于宗教学研究的术语,具有深刻的修辞和分析意图。
通过将商品与“物神”相联系,他意在揭示在以理性、科学自居的资本主义社会内部,存在着一种类似原始宗教崇拜的非理性现象。
资本主义宣扬其经济规律的客观性和科学性,但马克思观察到,在这个体系中,商品本身似乎被赋予了一种神秘的、超自然的力量。
正如在宗教拜物教中,
人创造出来的偶像或图腾反过来被当作具有独立意志和超凡能力的神物来顶礼膜拜,
在资本主义社会,人类劳动的产品——商品——也似乎获得了独立于其创造者的固有价值和自主运动的魔力。
这暗示着资本主义的运行,尽管其外表理性,实则深藏着一种颠倒的意识形态。