\section{核心内涵与理论意义}
商品拜物教是马克思用以剖析资本主义社会本质的关键概念之一。
它不仅描述了一种经济现象,更揭示了一种深刻的社会关系异化和意识形态迷雾。
\subsection{定义:社会关系的神秘化}
马克思将商品拜物教定义为:在以商品生产为基础的社会中,
人与人之间的社会生产关系采取了物与物之间关系的虚幻形式,
使得这些社会关系在人们面前表现为商品本身所固有的、超乎自然的属性。
简言之,就是“人手的产物”转变为似乎“赋有生命的、彼此发生关系并同人发生关系的独立存在的东西” 。
商品拜物教的核心在于“遮蔽”,即商品世界的物的关系掩盖了其背后的人的劳动的社会性质以及生产者之间的社会联系\autocite{zheng2022fetishism}。
\subsection{机制:物化与异化}
商品拜物教的形成机制与资本主义生产过程中的劳动异化和物化现象紧密相关。

首先,在商品交换中,生产商品的具体劳动和生产者本身往往是不可见的。
消费者在市场上购买商品时,通常只接触到商品本身及其价格,而生产这些商品所耗费的劳动时间、劳动条件以及劳动者的具体身份都被掩盖了。
商品的价值似乎是商品自身固有的,而不是产生于凝结在商品中的人类劳动\autocite{wiki}。

其次,价值被视为商品的内在属性。
经济价值,在商品拜物教的观念下,被心理上转化为(物化为)物品本身客观具有的属性。
人们相信商品天生就值那么多钱,而不是认识到其价值源于生产它的人类劳动和社会关系。

这种现象与马克思的“异化劳动”理论密切相关。
当劳动产品作为商品被生产出来,并被赋予了拜物教的神秘性之后,商品就成为一种同生产者相“异己”的东西,反过来支配着生产者自身。
在资本主义制度下,工人的劳动产品不属于工人,而属于资本家;
工人的劳动活动也非自主自愿,而是被迫的谋生手段。
生产者劳动的社会性不是在生产者自身那里直接体现,而是在流通领域中通过资本家的买卖行为间接体现。


\subsection{在马克思理论体系中的意义}
商品拜物教理论在马克思的整个思想体系中占据着举足轻重的地位。

\begin{enumerate}
    \item 它是马克思政治经济学的重要组成部分。

    马克思在《资本论》中正是从分析商品的二重性及其拜物教性质入手,逐步揭示资本主义生产方式的内在矛盾和运行规律的。不揭示商品的拜物教性质,就不可能完全了解商品的价值 。  

    \item 它与历史唯物主义紧密相连。

    商品拜物教并非人类社会的普遍现象,而是商品生产占统治地位的特定历史阶段(即资本主义社会)的产物。
    马克思通过将其与封建社会进行对比,指出在封建社会中,人身依附关系是社会的基础,
    劳动及其产品无需采取与其现实不同的虚幻形式,社会关系表现为个人关系,而未被伪装成物与物之间的社会关系。
    因此,商品的存在是历史的,拜物教是资本主义产物;商品生产被消灭了就没有拜物教,资本主义必将灭亡。

    \item 它深刻揭示了资本主义的剥削机制和意识形态功能。

    商品拜物教通过将商品的价值神秘化,掩盖了资本家对工人剩余价值的剥削。
    价值似乎源于商品本身或资本的魔力,而非工人的无偿劳动。
    这种观念上的颠倒,使得资本主义的剥削关系显得自然合理,从而起到了维护资本主义制度的意识形态作用。
    对商品拜物教的批判,就是要“釜底抽薪”式地揭露资本主义生产的秘密及其整个社会运动的矛盾和归宿\autocite{sxg}。
\end{enumerate}