除去语言方面的影响,乌克兰与俄罗斯在媒体,教育,
以及对特定历史事件的解读方面也有着较大的分歧,这些分歧导致俄乌关系恶化,引发俄乌冲突。

\subsection{对历史事件的不同解读}
乌克兰的俄语社区在一些重大历史事件上的看法与乌克兰语社区不同。
亲俄的乌克兰人往往认同俄罗斯的历史视角,认为俄乌同属于“兄弟民族”;
而乌克兰语社区则强调与俄罗斯的区别,认同独立的乌克兰历史,尤其重视摆脱苏联统治的历史。

两国对于苏联解体的看法就有着巨大的分歧。

苏联解体是影响俄乌关系的重要历史事件。对于俄罗斯来说,苏联的解体是一个巨大的政治和文化创伤,
许多俄罗斯人认为,失去对乌克兰等加盟共和国的控制意味着俄罗斯失去了在世界舞台上的重要地位和影响力。
俄罗斯常常把苏联时期的强大视为一种荣耀,认为其国家地位与历史使命在世界范围内得到了认可。

对于乌克兰人而言,苏联时期的统治常常被视为压迫和不公,
尤其是大饥荒(霍洛多莫尔)等历史事件。

\subsection{媒体和教育的影响}
俄罗斯通过媒体、文化、教育等手段在乌克兰东部和南部地区影响俄语社区,塑造亲俄的历史观和身份认同。

与此相对,乌克兰政府则在推动去俄罗斯化教育,以强化乌克兰的独立性。
这种文化对立加剧了不同语言社群的身份割裂。