\documentclass{source/Paper}
\usepackage{tikz}
\usetikzlibrary{graphs, positioning, quotes, shapes.geometric}
\articletitle{俄乌冲突溯因}
\name{何铭源}
\stuid{3240104481}
\Abstract{
    俄乌冲突是近年来影响深远的地区性战争,尽管俄罗斯与乌克兰同为斯拉夫民族,最终却走向兵戎相见。
    本文将从历史背景、文化和政治因素等方面分析俄乌冲突爆发的根源,
    深入探讨其成因,并总结该冲突对全球和平的启示,强调国际秩序维护、和平解决争端的重要性,以及世界各国在面对地区冲突时的责任和协作需求。
}
\Keyword{战争;俄罗斯;乌克兰;俄乌冲突;俄乌关系}
\engarticletitle{The Reason behind the Russo-Ukrainian Conflict}
\engname{He ming-yuan}
\engKeyword{War; Russia; Ukraine; Russo-Ukrainian Conflict; Russo-Ukrainian Relationship}
\engAbstract{
    The Russia-Ukraine conflict is one of the most far-reaching regional wars in recent years.
    Although Russia and Ukraine share a Slavic heritage, they have ultimately come to military confrontation.
    This paper analyzes the root causes of the conflict, examining its historical, cultural, and political factors, and explores the broader implications of this conflict for global peace. It emphasizes the importance of upholding international order, resolving disputes peacefully,
    and the need for international cooperation and responsibility in addressing regional conflicts.
}
\begin{document}
\maketitles
\newpage
\tableofcontents
\newpage
\setcounter{section}{-1}
\section{引言}
俄乌冲突始于2014年克里米亚事件,2022年升级为全面战争。
在这个冲突中,俄罗斯以“去军事化”,“去纳粹化”乌克兰为由,发动“特别军事行动”,随后乌克兰全境战事不断,双方伤亡惨重。
在这个持久而又惨烈的战争中,本论文想要探究俄乌战争所发生的原因,以及一些关于如何取得和平的启示。
\section{历史背景}
\subsection{苏联解体后乌克兰的身份认同问题}
苏联解体与乌克兰独立在俄乌冲突的原因中扮演了基础性和催化性的角色,为之后的矛盾埋下了伏笔。
俄罗斯与乌克兰曾均属于苏联,而在苏联解体后,乌克兰成为独立国家,使得乌克兰出现了身份认同问题,其中表现在语言,以及北约东扩的问题中。
\subsubsection{俄语对于俄乌冲突的影响}
1922年苏联成立,在苏联的统治下,俄语成为通用语言,而乌克兰原有的乌克兰语遭到了压制与边缘化的待遇。

苏联解体后,国际形势的大变革给乌克兰带来了深远的影响,这也使得乌克兰对于母语认同的态度也愈发强烈。

俄语地位在乌克兰国内也引起了许多争端,造成了第二国家语言之争,第二官方语言之争等冲突。\cite{__2023-1}

在语言立法方面,
乌克兰政府近年来推动多项语言立法,意图提升乌克兰语的地位,
乌克兰官方规定乌克兰语为唯一官方语言,并在教育、公共事务中使用。
这些举措引发了俄语使用者的不满,认为这些政策威胁了他们的语言和文化权利。

在乌克兰的亲俄地区,这些语言政策遭到反对,
加剧了这些地区的分离倾向,成为俄罗斯干涉乌克兰事务的借口之一。
俄罗斯政府以保护“同胞”的语言权利为由,
声称乌克兰的语言政策“歧视”俄语使用者,为其干预提供了正当性。

\subsubsection{地域差异}
乌克兰的语言分布呈现明显的地域差异。东部和南部地区的居民多讲俄语,文化上更接近俄罗斯;
而西部和中部地区主要使用乌克兰语,偏向欧洲。这种语言分布导致了不同地域对乌克兰国家身份的认同差异。

\subsection{俄罗斯与乌克兰关系的演变}
苏联时期,乌克兰作为加盟共和国之一在经济、政治上受到莫斯科的直接控制,形成了深刻的依附关系。
然而,1991年苏联解体后,乌克兰独立并逐渐寻求摆脱俄罗斯的影响,尤其是尝试融入欧洲和北约,以保障自身安全并加强与西方的经济合作。

在这段时期,俄罗斯对乌克兰的影响力逐渐减弱,并在颜色革命,克里米亚事件后,两者的关系进入低谷。

近年,两国关系更趋紧张,俄乌冲突演变为持续的军事对抗,不仅涉及领土问题,更深层次地反映了民族认同、历史记忆及地缘政治对抗的矛盾。
\begin{figure}[htbp]
    \centering
    \begin{tikzpicture}
        \node[draw](start){同属东斯拉夫文化圈};
        \node[draw, below=of start](next1){苏联成立,乌克兰成为加盟共和国被苏联中央所控制};
        \node[draw, below=of next1](next2){苏联解体,乌克兰逐渐受到西方国家的影响,俄罗斯的控制力减弱};
        \node[draw, below=of next2](next3){在一系列历史事件后两国关系走向破裂};
        \graph{
            (start) -> (next1) -> (next2) -> (next3)
        };
    \end{tikzpicture}
    \caption{俄乌关系演变}
\end{figure}
\subsection{北约东扩}
北约东扩,是冷战后欧洲安全格局深刻重塑的标志。
随着苏联的解体,许多东欧国家选择摆脱过去的东方阵营束缚,
争取加入西方主导的军事联盟以寻求新的安全保障。

北约的成员国逐步从最初的西欧国家扩展到包括波兰、匈牙利、捷克、波罗的海三国等在内的东欧国家,并向俄罗斯的西部边界逼近。
对这些新成员而言,加入北约不仅意味着获得军事保护,更是融入西方政治经济体系的途径。

然而,对俄罗斯来说,北约的扩张被视为对其安全利益的威胁。俄罗斯历来将东欧视为战略缓冲区,北约的东扩在其眼中是对冷战后“不东扩”承诺的背弃,更是对自身地缘安全的侵蚀。这一进程中的核心问题不仅仅是安全感的缺失,更是俄罗斯对西方主导的全球秩序的不满和对自身影响力的重申。

因此,北约的不断扩展逐步激化了俄欧关系,为俄乌冲突埋下了导火索,使得乌克兰作为两大阵营的“夹缝国家”陷入了地缘政治的漩涡之中。

美国在乌克兰建交初期,就积极将乌克兰拉入西方阵营,并在乌克兰发动颜色革命,推动乌克兰“入约”进程。乌克兰方面自身也在积极向北约靠拢,积极开展联合军事演练。\cite{_19922020_2023}这些举措同样引起俄罗斯方面的不满。
\section{文化因素}
除去语言方面的影响,乌克兰与俄罗斯在媒体,教育,
以及对特定历史事件的解读方面也有着较大的分歧,这些分歧导致俄乌关系恶化,引发俄乌冲突。

\subsection{对历史事件的不同解读}
乌克兰的俄语社区在一些重大历史事件上的看法与乌克兰语社区不同。
亲俄的乌克兰人往往认同俄罗斯的历史视角,认为俄乌同属于“兄弟民族”;
而乌克兰语社区则强调与俄罗斯的区别,认同独立的乌克兰历史,尤其重视摆脱苏联统治的历史。

两国对于苏联解体的看法就有着巨大的分歧。

苏联解体是影响俄乌关系的重要历史事件。对于俄罗斯来说,苏联的解体是一个巨大的政治和文化创伤,
许多俄罗斯人认为,失去对乌克兰等加盟共和国的控制意味着俄罗斯失去了在世界舞台上的重要地位和影响力。
俄罗斯常常把苏联时期的强大视为一种荣耀,认为其国家地位与历史使命在世界范围内得到了认可。

对于乌克兰人而言,苏联时期的统治常常被视为压迫和不公,
尤其是大饥荒(霍洛多莫尔)等历史事件。

\subsection{媒体和教育的影响}
俄罗斯通过媒体、文化、教育等手段在乌克兰东部和南部地区影响俄语社区,塑造亲俄的历史观和身份认同。

与此相对,乌克兰政府则在推动去俄罗斯化教育,以强化乌克兰的独立性。
这种文化对立加剧了不同语言社群的身份割裂。
% \section{地缘政治因素}
% \section{经济因素}
\section{近期导火索}
\subsection{克里米亚事件}
俄罗斯联邦并吞克里米亚指的是俄罗斯于2014年2月20日至3月18日入侵乌克兰,
占领并径自宣布吞并克里米亚半岛,
以及其后在乌克兰东南方的行动所引发的一连串政治风波。\cite{noauthor__2024}

克里米亚事件对俄乌关系及全球格局产生了深远影响。
2014年俄罗斯吞并克里米亚,导致乌克兰失去重要战略半岛及资源,激发了乌克兰国内的民族主义情绪,加深了亲欧势力的主导地位。

此次事件阻碍了俄罗斯国内的发展进程,并且冲击着当前的国际格局。\cite{__2023-2}

乌克兰更加坚定地寻求与西方的合作,政治分裂进一步加剧。
对俄罗斯而言,吞并克里米亚被视为恢复历史正义,但同时遭遇西方国家的经济制裁,严重影响其经济发展。

国际上,俄罗斯的行为被广泛谴责,制裁未能迫使俄罗斯撤回,反而加剧了与西方的对立,推动了冷战后紧张局势的重现。
\subsection{顿巴斯战争}
顿巴斯战争是自2014年起在乌克兰东部顿涅茨克和卢甘斯克地区爆发的冲突。
冲突源于乌克兰亲西方革命后的政治变化,部分俄罗斯裔居民在当地宣称独立,并寻求加入俄罗斯。
乌克兰政府视此为分裂活动,派遣军队进行镇压。
亲俄武装得到俄罗斯的军事支持,战斗迅速升级,造成大量伤亡和人道危机。
尽管2015年达成“明斯克协议”停火,战斗仍时断时续,局势持续紧张,成为俄乌关系的关键问题,并为后来的全面战争埋下伏笔。

顿巴斯战争加深了乌克兰的民族认同感,使乌克兰更坚定地走向与西方的对接,尤其是在经济和安全方面。
乌克兰政府在战争中获取了西方国家的援助,同时也坚定了拒绝俄罗斯影响力的立场。

而俄罗斯在这次战争中的介入不仅使得冲突变得更加激烈,也使其与乌克兰的对抗从政治和外交层面进一步升温,推动了两国之间的长期敌对关系。

\subsection{总结}
\begin{figure}[htbp]
    \centering
    \begin{tikzpicture}[node distance=10pt]
    \node[draw] (start) {乌克兰自1991年苏联解体后独立,但其政治和文化仍然受到俄罗斯的深刻影响};
    \node[draw, below=of start] (next1) {2014年,乌克兰爆发亲欧盟的“广场革命”,克里米亚并入俄罗斯成为冲突的导火索};
    \node[draw, below=of next1] (next2) {2014年2月起,在乌克兰东部和南部发生顿巴斯战争};
    \node[draw, below=of next2] (next3) {2022年,俄罗斯在“特殊军事行动”名义下对乌克兰发动全面进攻,标志着俄乌冲突进入新阶段};
    \graph{
    (start) -> (next1) -> (next2) -> (next3)
    };
    \end{tikzpicture}
    \caption{俄乌冲突的演变过程}
\end{figure}
\section{外部因素}
% \subsection{外部因素}
\begin{enumerate}
    \item 西方国家的援助和军事支持:美国、欧盟等对乌克兰的支持
    \item 中国的中立与国际力量博弈:俄乌冲突下的全球政治格局变化
    \item 冲突的全球性影响:能源危机、粮食危机等国际连锁反应
\end{enumerate}
\section{结论}
俄乌冲突既有俄罗斯,乌克兰历史不可调和的矛盾,以及所带来的文化,政治的冲突。
也有西方国家的推波助澜,使得冲突扩大演变为战争。

此次矛盾的核心在于:
\begin{enumerate}
    \item 乌克兰的北约成员国地位问题。
    \item 克里米亚的领土纠纷及乌东地区的独立问题。
    \item 俄罗斯的安全诉求问题。
\end{enumerate}

总体来看,俄乌冲突为世界带来了警示,强调了各国维护和平、坚持国际法和加强外交合作的重要性。

首先,它凸显了地区冲突在全球化时代的深远影响。俄乌冲突不仅局限于两国之间的领土争端,还迅速演变为大国博弈的舞台,导致全球能源和粮食危机,影响了许多国家的经济稳定。

其次,俄乌冲突揭示了国际秩序和规则的重要性。冲突中对主权的侵害、国际法的挑战,以及各国不同的应对措施表明,国际规则的执行和各国对秩序的尊重是维护和平的关键。

此外,俄乌冲突凸显了外交和和平解决争端的必要性。冲突的迅速升级显示出军事手段并不能有效解决深层次的历史和民族矛盾,反而加剧了敌对关系和长期的地区不稳定。和平谈判、外交手段以及对话成为预防战争的最佳途径,避免局部冲突扩大为更严重的全球危机。
\bibliography{ref}
\newpage
\appendix
\section{附-对课程的感想}
\subsection{写作感想}
在完成这个作业之前,我几乎不怎么关注国际形势,俄乌冲突虽然已经持续了很长时间,但是由于个人兴趣和高三学业等原因。我一直没能深入了解本次事件。
这次的小论文给我一个很好的机会,让我去广泛的搜集资料和文献,并从多个角度去探寻俄乌冲突的起源。

俄乌冲突不仅起源复杂,而且对国际格局也有很大的冲击,算得上是本世纪的一件大事情。

这次冲突看起来最主要的原因在于乌克兰亲近西方国家,受到了北约西扩的影响,但是深究来看,其历史因素和民族因素甚至语言文字和地缘政治都在本次冲突中
发挥了不一般的作用。而众多复杂的因素才使得本次冲突发生。

感谢军理课给了我一次完成小论文的机会,而这也是我在大学阶段中完成的第一篇小论文。从资料搜集,到内容产出,再到排版及文献引用,诸多过程都是我
从未经历过的,而我也在这个过程中,边做边学,积累了许多宝贵的经验。

\subsection{对军理课的感想}
程老师的课程十分生动有趣,课上讲到的很多观点也会有许多详实的例子,虽然我还未修读史纲,高中时也没有选修历史,但是我仍然听的津津有味。在军理课上,我感觉我学到很多,
从国防教育,到战争,这些看似与我们生活很远的事情其实与我们息息相关。

军理课培养了我的爱国情怀,以及让我更加关注国际形式的变化。
\end{document}