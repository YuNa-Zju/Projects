\documentclass[math]{amznotes}
\usepackage{pdfpages}
\usepackage{ctex}
\usepackage{float}
\usepackage{caption}
\usepackage{enumitem}
\usepackage{booktabs}
\usepackage{tabularx}
\usepackage{tikz}
\usepackage[svgnames]{xcolor}
\usetikzlibrary{mindmap,trees,shadows,positioning, arrows.meta}
% \usepackage{amssymb}     % 提供 \checkmark
\usepackage{lipsum}     % filler text
%dfnbox exbox tecbox notebox lembox(引理) thmbox
\begin{document}
\setlength{\parindent}{2em}
\setcounter{tocdepth}{2} % 设置目录深度为2
\tableofcontents
\newpage
\chapter{线性微分方程}
\begin{dfnbox}{线性微分方程}{linear-differential-equation}
    线性微分方程是指形如
    \begin{equation*}
        a_n(x)y^{(n)} + a_{n-1}(x)y^{(n-1)} + \cdots + a_1(x)y' + a_0(x)y = f(x)
    \end{equation*}
    的微分方程,其中 $a_i(x)$ 和 $f(x)$ 是已知函数,$y$ 是未知函数。
\end{dfnbox}
其中$f(x)$被称为非齐次项,当$f(x) = 0$时,方程称为齐次线性微分方程。
此时引入算子$L[y] = a_n(x)y^{(n)} + a_{n-1}(x)y^{(n-1)} + \cdots + a_1(x)y' + a_0(x)y$,则线性微分方程可以写为
\begin{equation*}
    L[y] = f(x)
\end{equation*}
\section{齐次线性方程}
若$L[y] = 0$,则$y = 0$是方程的一个解。

若\begin{equation*}
  \left\{
  \begin{split}
  L[y] = 0\\
  y^{(i)}(x) = 0
  \end{split}
  \right.
\end{equation*}
则齐次线性微分方程存在唯一零解。
\begin{tecbox}{$L$算子具有线性性}{linear-property-of-L}
    \begin{eqnarray*}
      L[c_1y_1 + c_2y_2] = c_1L[y_1] + c_2L[y_2]
    \end{eqnarray*}
\end{tecbox}
\begin{dfnbox}{朗斯基行列式}{Wronskian}
    设$y_1, y_2, \ldots, y_n$是$n$个可微函数,则它们的朗斯基行列式定义为
    \begin{equation*}
        \omega(y_1, y_2, \ldots, y_n) =
        \begin{vmatrix}
            y_1 & y_2 & \cdots & y_n \\
            y_1' & y_2' & \cdots & y_n' \\
            \vdots & \vdots & \ddots & \vdots \\
            y_1^{(n-1)} & y_2^{(n-1)} & \cdots & y_n^{(n-1)}
        \end{vmatrix}
    \end{equation*}
\end{dfnbox}
可以使用朗斯基行列式判断解是否线性相关。
若在区间$[a, b]$内,$\omega(y_1, y_2, \ldots, y_n) \neq 0$,则$y_1, y_2, \ldots, y_n$线性无关;若$\omega(y_1, y_2, \ldots, y_n) = 0$,则$y_1, y_2, \ldots, y_n$线性相关。
\begin{notebox}
朗斯基行列式要么恒为0,要么恒不为0。
\end{notebox}
\begin{dfnbox}{基本解组}{fundamental-solution-set}
    $n$阶齐次线性微分方程的``解空间''维数为$n$维,因此
    设$y_1, y_2, \ldots, y_n$是齐次线性微分方程的$n$个线性无关解,则称它们为该方程的基本解组。
\end{dfnbox}
\section{非齐次线性方程}
非齐次线性方程解的结构为齐次线性方程的通解+一个特解。
\begin{equation*}
  y = Y + y^*
\end{equation*}

所以想要求解非齐次线性微分方程就是求出他的齐次通解,再求出特解。

若\begin{equation*}
  L[y] = f(x) = f_1(x) + f_2(x)
\end{equation*}

如果$f(x)$不好处理,但可以分拆为$f_1(x)$与$f_2(x)$,使得$L[y] = f_1(x)$与$L[y] = f_2(x)$都容易处理。
那么$f(x)$对应的$y^* = y_1^* + y_2^*$。

针对复数\begin{equation*}
  L[y] = f(x) = U(x) + \mathrm{i} V(x)
\end{equation*}

则\begin{gather*}
  L[u] = U(x) \\
  L[v] = V(x)
\end{gather*}
\section{常系数\textbf{线性}微分方程}
\begin{dfnbox}
{常系数线性微分方程}{constant-coefficient-linear-differential-equation}
    常系数线性微分方程是指形如
\begin{equation*}
    \dfrac{\mathrm{d}^n y}{\mathrm{d}x^n} + p_1(x) \cdot \dfrac{\mathrm{d}^{n-1} y}{\mathrm{d}x^{n-1}} + \cdots + p_{n - 1}(x) \cdot \dfrac{\mathrm{d} y}{\mathrm{d}x} + p_n(x) \cdot y = f(x)
\end{equation*}

当$p_1, p_2, \ldots, p_n$为常数时,称为常系数线性微分方程。
\end{dfnbox}
\subsection{齐次常系数线性微分方程的解法}
以二阶常系数线性微分方程为例:
\begin{equation*}
    y'' + py' + qy = 0
\end{equation*}

写出其特征方程
\begin{equation*}
    \lambda^2 + p\lambda + q = 0
\end{equation*}

则该方程的解为
\begin{equation*}
  y = \left\{
    \begin{aligned}
      \;&c_1 \mathrm{e}^{\lambda_1} + c_2 e^{\lambda_2} \qquad \Delta \ge 0 \quad\text{特征方程有两个不等实根}\\
      &(c_1x + c_2) \mathrm{e}^{\lambda} \qquad \Delta = 0 \quad \text{特征方程有一个实根}\\
      &c_1 \mathrm{e}^{\alpha x} \cos(\beta x) + c_2 \mathrm{e}^{\alpha x} \sin(\beta x) \qquad \Delta < 0 \quad \text{特征方程有两个共轭复根}\lambda = \alpha \pm \mathrm{i}\beta\\
    \end{aligned}
  \right.
\end{equation*}
\begin{exbox}{两个实根}{}
    设$y'' + 3y' + 2y = 0$,则特征方程为
    \begin{equation*}
        \lambda^2 + 3\lambda + 2 = 0
    \end{equation*}

    解得$\lambda_1 = -1, \lambda_2 = -2$,因此通解为
    \begin{equation*}
        y = c_1 e^{-2x} + c_2 e^{-x}
    \end{equation*}
\end{exbox}
\begin{exbox}{两个重根}{}
    设$y'' + 2y' + y = 0$,则特征方程为
    \begin{equation*}
        \lambda^2 + 2\lambda + 1 = 0
    \end{equation*}

    解得$\lambda_1 = \lambda_2 = -1$,因此通解为
    \begin{equation*}
        y = (c_1 + c_2 x) e^{-x}
    \end{equation*}
\end{exbox}
\begin{exbox}{共轭负根}{}
    设$y'' + y' + y = 0$,则特征方程为
    \begin{equation*}
        \lambda^2 + \lambda + 1 = 0
    \end{equation*}

    解得$\lambda_1 = \frac{-1 + \sqrt{3}i}{2}, \lambda_2 = \frac{-1 - \sqrt{3}i}{2}$,因此通解为
    \begin{equation*}
        y = e^{-\frac{x}{2}}(c_1 \cos(\frac{\sqrt{3}}{2}x) + c_2 \sin(\frac{\sqrt{3}}{2}x))
    \end{equation*}
\end{exbox}
总结一下:
\begin{equation*}
  y^n + p_1 y^{n-1} + p_2 y^{n-2} + \cdots + p_{n-1} y' + p_n y = 0
\end{equation*}
的特征方程为$D(\lambda)$且有$s$个互异的实根,$\lambda_1$为$n_1$重根,$\lambda_2$为$n_2$重根,$\ldots$,$\lambda_s$为$n_s$重根,则通解为
\begin{equation*}
  \begin{aligned}
    &y_1 = c_1 \cdot e^{\lambda_1 x}\\
    &y_2 = c_2 \cdot e^{\lambda_2 x}x\\
    &\vdots\\
    &y_{s} = e^{\lambda_s x}\\
    &\lambda_j = \alpha + \mathrm{i}\beta \text{且是}n_j\text{重根}\\
    &y_j = (c_1 + c_2 x + \cdots + c_{n_j} x^{n_j - 1}) e^{\alpha x}(\cos(\beta x) + \mathrm{i} \sin(\beta x))
    &y = y_1 + y_2 + \cdots + y_s
  \end{aligned}
\end{equation*}
\section{非齐次常系数线性微分方程}
以二阶常系数线性微分方程为例:
\begin{equation*}
    y'' + py' + qy = f(x)
\end{equation*}
\subsection{f为多项式与e指数相乘}
若$$f(x) = P_n(x)\mathrm{e}^{\alpha x}$$

那么
\begin{equation*}
    y^* = x^k R_n(x)\mathrm{e}^{\alpha x}
\end{equation*}
其中$R_n(x)$是一个$n$次多项式,$k$是满足以下条件的最小非负整数:
其中\begin{equation*}
  k = \left\{
    \begin{aligned}
      \;&0 \qquad&\text{若特征方程的根均不为}\alpha\\
      &1 \qquad&\text{若特征方程的根中有一个为}\alpha\\
      &2 \qquad&\text{若特征方程的根中为二重根且为}\alpha\\
    \end{aligned}
  \right.
\end{equation*}
\begin{exbox}{非齐次常系数线性微分方程}{}
    设$y'' - 2y' - 3y = (x - 5)e^{-x}$,则特征方程为
    \begin{equation*}
        \lambda^2 - 2\lambda - 3 = 0
    \end{equation*}
    解得$\lambda_1 = 3, \lambda_2 = -1$,因此齐次方程的通解为
    \begin{equation*}
        Y = c_1 e^{3x} + c_2 e^{-x}
    \end{equation*}
    对于非齐次项$(x - 5)e^{-x}$,由于特征方程的根中有一个为$-1$,因此$k = 1$,所以特解为
    \begin{equation*}
        y^* = x R(x) \mathrm{e}^{-x} = (Ax + B) \mathrm{e}^{-x}
    \end{equation*}
    将其代入原方程求解系数$A, B$。
\end{exbox}
\subsection{f为多项式与e指数相乘再与三角函数相乘}
对于$$f(x) = P_n(x)\mathrm{e}^{\alpha n}(\cos \beta x)$$

或
$$f(x) = P_n(x)\mathrm{e}^{\alpha n}(\sin \beta x)$$

则构造$$F(x)= P_n(x)\mathrm{e}^{(\alpha + \mathrm{i}\beta)x}$$
那么$$f(x) = \left\{
\begin{aligned}
  \;&\mathrm{Re}(F(x))\qquad \text{三角函数为\textrm{cos}}\\
  &\mathrm{Im}(F(x))\qquad \text{三角函数为\textrm{sin}}
\end{aligned}
\right.
$$。
\begin{exbox}{f(x)带有三角函数}{}
\begin{equation*}
\begin{aligned}
  y'' + a^2y &= \cos \alpha x\\
  \lambda &= \pm \mathrm{i}a\\
  Y &= c_1 \cos(ax) + c_2 \sin(ax)\\
  \text{构造}y'' + a^2y &= \mathrm{e}^{\mathrm{i}\alpha x} \\
  y'^* &= xA\mathrm{e}^{\alpha x \mathrm{i}}\\
  \text{真正的}y^* &= \mathrm{Re}(y'^*)
\end{aligned}
\end{equation*}
\end{exbox}
\section{变系数方程}
\subsection{欧拉方程}
\begin{dfnbox}{欧拉方程}{Euler-equation}
    欧拉方程是指形如
    \begin{equation*}
        a_0x^ny^{(n)} + a_1x^{n-1}y^{(n-1)} + \cdots + a_{n-1}x^2y' + a_nx^1y = f(x)
    \end{equation*}
    的微分方程,其中$a_0, a_1, \cdots, a_n$为常数。
\end{dfnbox}

此时做变量代换$x = e^t$,则
\begin{align*}
    \dfrac{\mathrm{d}y}{\mathrm{d}x} = \dfrac{\mathrm{d}y}{\mathrm{d}t}\cdot \dfrac{\mathrm{d}t}{\mathrm{d}x} = \dfrac{\mathrm{d}y}{\mathrm{d}t}\cdot \dfrac{1}{x}\\
    \dfrac{\mathrm{d}^2y}{\mathrm{d}x^2} = \dfrac{\mathrm{d}^2y}{\mathrm{d}t^2}\cdot \left(\dfrac{\mathrm{d}t}{\mathrm{d}x}\right)^2 = \dfrac{\mathrm{d}^2y}{\mathrm{d}t^2}\cdot \dfrac{1}{x^2}
\end{align*}

以二阶为例:
\begin{equation*}
    a_0x^2y'' + a_1xy' + a_2y = f(x)
\end{equation*}

变为
\begin{equation*}
    a_0y'' + (a_1 - a_0)y' + a_2y = f(e^t)\text{或}f(e^{-t})
\end{equation*}
\begin{exbox}{欧拉方程}{}
    设$x^2y'' - 4xy' + 6y = x + x^2\ln x$,令$x = e^t$,则原式变为
    \begin{align*}
        \dfrac{\mathrm{d}^2 y}{\mathrm{d}t^2} - 5\dfrac{\mathrm{d}y}{\mathrm{d}t} + 6y &= e^t + t \cdot e^{2t}\\
        Y &= c_1 \mathrm{e}^{3t} + c_2 \mathrm{e}^{2t}\\
        y^* &= A\mathrm{e}^{t} + t(Bt + C)\mathrm{e}^{2t}\\
        \text{最后一步:}t &= \ln x\\
    \end{align*}
\end{exbox}
\end{document}