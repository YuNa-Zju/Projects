\subsection{分工}
\begin{table}[H]
    \centering
    \caption{分工表}
    \begin{tabularx}{\textwidth}{>{\raggedright\arraybackslash}X >{\raggedright\arraybackslash}X}
      \toprule
      姓名  &   分工\\
      \midrule
        何铭源 &    确立项目,统筹策划,答辩,PPT修改,参与制定问卷并帮助投放,活动策划并组织,SQTP活动册整理和撰写\\
        严骏驰 &    采访及数据分析,辅助组织统筹,问卷投放,整理采访内容并对采访内容进行了分析,整理得出假新闻的识别与应对的一系列结论。\\
        何天佑 &    部分问卷设计,采访,活动海报设计,活动部分问题编写\\
        吴恩迪 &    推文制作,学习类资料搜集技巧\\
        池昊璞 &    部分问卷设计,部分问卷制作,投放与整理,搜索资料提出问题\\
        郑宇杰 &    11月开题答辩PPT制作,问卷投放,现场活动问题撰写,部分答辩PPT制作\\
        王弘晰 &    前期部分问卷设计,中期物资采购,后期资料整理(生活类部分),预约线下场地\\
        王俊凯 &    部分问卷设计,问卷制作,投放与整理,导出并分析问卷结果,整理知识点\\
        李泽楷 &    关于学生搜索现状的资料搜集,总结“理工类和社科类开题立项前的文献调研报告”\\
        唐焯奇 &    部分问卷设计,初期资料收集,负责总结信息收集整合相关方法 \\
      \bottomrule
    \end{tabularx}
  \end{table}
\subsection{感想}
何铭源:
\begin{notebox}
  从高中思维到大学思维的转变,对许多同学而言都是一道坎。小组作业中,总有部分同学习惯性地索要资料,而互联网上信息唾手可得的现状,更凸显了独立信息收集能力的重要性。因此,我们的SQTP项目聚焦于“信息收集与整合能力的培养”。

活动分为前期与后期。前期,我们广泛搜集相关资料,为后续工作指明方向。后期,我们创新性地借鉴游戏机制,旨在帮助同学们更快地接受“先搜索,再提问”的逻辑,并将我们整理出的信息收集方法融入日常学习。

作为项目的立项人,我肩负着活动策划和小组统筹的双重责任,这对我来说是一次极大的考验。完成整个SQTP后,我对信息收集,尤其是文献资料的搜集有了更深刻的理解。同时,在项目管理和活动策划方面,我的能力也得到了显著提升。
\end{notebox}
吴恩迪:
\begin{notebox}
  本次SQTP项目,围绕提升大学生搜索信息的能力展开,前期,我们进行了问卷投放、采访同学等工作,进行了初步的对同学们搜索信息的能力以及状况的了解,而后期,我们也根据获取的初步情况进行了活动的进一步展开。故而,在线下活动的展开之中,我们更加进一步的帮助同学们了解了更多所搜信息的渠道,全方面的对待这些渠道。此次活动的展开,不仅让我体会到了团队合作的魅力,也让我完善了信息搜索能力的方法知识,进一步提升了自我的能力品质。
\end{notebox}
王弘晰:
\begin{notebox}
  本次SQTP项目不知不觉走向了尾声。我为能参与其中,作出贡献感到自豪。

  此次项目聚焦于大一新生信息检索能力提升,我认为是非常贴合实际,有着巨大意义的。因为我自己检索能力也不突出,所以在项目的推进过程中,我也借此机会进行了广泛的调查和学习,在提升个人能力的同时也能够通过我们整理的资料去帮助身边的同学,这对于营造积极的学风与学习氛围有着正向影响,我想着也是学校组织,鼓励我们做SQTP项目的意义,也即在个人发展的同时促进集体的综合素养提升。

  值得一提的是,此次SQTP项目的目的和意义也与我作为班级学习委员的职责和使命高度吻合,也让我在工作过程中不断思考感悟,从而以更饱满的精神状态投入到学习和工作中。

  未来,我也将继续消化和运用本次项目中各成员做出的整理和分析,让这些技巧实打实地为我们的学习生活做出积极贡献。同时,项目的结题也不是为我们停止学习和推广画上句号,我们会吸取经验,并继续在相关领域做出努力和贡献。
\end{notebox}
李泽楷:
\begin{notebox}
  在完成小组调研任务的同时,我自己本身通过直播讲解获得了许多关于学术论文搜索的知识,
  并对学校图书馆的数据库、知网和搜索型AI等有了更多的了解,这会帮助我在以后的科研任务中更快速准确地获得我想要的资料,
  而这也是我想通过这次SQTP活动帮助同学们get到的技能,
  让同学们能更早更好地成为信息搜集者。
\end{notebox}
何天佑:
\begin{notebox}
  本次SQTP项目所调查的主题是大学生信息搜集方面的能力,这恰好是我在刚进入大学所欠缺的能力,
  因此,通过本次SQTP项目的进展,我不仅了解到大学生信息搜索能力的现状,而且还从中学到了很多获取信息的途径,
  不管是在日常生活上还是在学业科研方面,甚至是在休闲娱乐方面都让我有了很深的了解和经验,以及在项目进行过程中,
  结实了很多志同道合的伙伴,大家在一起工作的过程中互相帮助互相包容,锻炼了我团队协作能力和沟通交流的能力。
  总而言之,这次项目经历让我受益匪浅,希望可以与同学们一起参加更多的项目。
\end{notebox}
郑宇杰:
\begin{notebox}
  \begin{enumerate}
    \item 提问时要结合自身生活实际。问题涵盖面要广,体现出主题的同时也要保证一定的趣味性。
    \item 善于利用搜索工具和网络社区,寻找更广泛人群更普遍的问题。
  \end{enumerate}
\end{notebox}
严骏驰:
\begin{notebox}
  我们组聚焦于信息检索能力提升,整合了生活学习科研等多方面的信息搜集技巧。
  并通过线下分享会等形式传播我们的研究成果,既帮助大家填充了一些检索领域的空白,又帮助我们自身提升综合检索能力,受益匪浅。
\end{notebox}
池昊璞:
\begin{notebox}
  参与项目前,我以为信息检索不过是输入关键词的简单操作,直到真正设计采访提纲时,才意识到系统性检索的难度——如何精准定位冷门资料、辨别信息真伪都是学问。后期搜索资料设计问卷过程中也暴露了问题,比如我曾误抄一整段的AI文献,但后来发现直接大段摘录简直八竿子打不着。最触动的还是后面团队互助、经验传承的细节:感觉什么都没干时,何天佑为我指点迷津,差点忘记发感悟时,严骏驰委婉的和我讲,被检索式困住时,也有学长提醒“用限定词缩小范围”,这些零散的经验传递比教程更有效。唯一遗憾的是,许多实践中摸索出的非常规检索路径未被系统记录,若能整理成案例库或许更有价值。
\end{notebox}