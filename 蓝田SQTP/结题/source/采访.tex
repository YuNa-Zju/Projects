在这个信息过载的时代,检索能力已成为核心学术竞争力之一。
我们的采访所揭示的,不仅是个体的困惑,更是整个教育体系需要应对的时代命题。
我们不仅需要专业知识,更需要在大浪淘沙的信息海洋中辨别真伪、提取精华的能力。

\subsection{实操困境}
当被问及课程论文写作的第一反应时,学术数据库成为首选,这反应出我们的学生虽然知道应该使用正规学术渠道,却在实操层面面临诸多困境:
关键词搜索成为横亘在学术探索道路上的一道障碍。
"不确定自己搜的关键词对不对"、
"专业术语让人迷惑"等反馈\dots

学术数据库的检索逻辑与日常搜索引擎大相径庭、
但鲜有课程系统教学这一关键技能。
更值得玩味的是,受访者提到的"把中文关键词机翻成英文再搜索"的小技巧,
既展现了学生的创造性,也折射出工具使用教育的不足——这本应是基础技能,却成了需要自行摸索的"技巧"。

\subsection{信息渠道多元化}
值得指出的是,信息获取渠道的多元化趋势在采访中尤为明显。
短视频平台、校园论坛、经验分享会与传统学术数据库共同构成了学生的信息来源。
学生对学长学姐经验的重视反映出学术界一个长久存在的困境:隐性知识的传承往往依赖于非正式渠道。
同时,AI工具的崛起正在重塑信息检索的生态。学生对"用AI工具快速筛选文献"的强烈需求,既是对效率的追求,也隐含对传统检索方式的不满。
但关键在于,我们需要明白如何与AI协作而非依赖,如何用批判性思维审视AI提供的结果,而非全盘接受。

