\subsection{采访刘艳老师}
问:您对``容貌焦虑''有哪些了解?

答:``容貌焦虑''并非明确的心理或精神障碍诊断概念,只是社会上的流行说法。它涵盖了人们对外在形象是否符合某种标准的担忧,可实际上根本没有这样的标准。在真实生活中,相处久了,大家不会仅凭容貌定义一个人。但社会上存在一种认知,觉得漂亮很重要,似乎要达到某个不知从何而来的标准。

我遇到过一些情况,有人觉得必须整容,否则大学都读不下去,甚至日子都过不下去。这种``容貌焦虑''是一种外在表现,是担心容貌不够好而失去某些东西。
社会心理学对此有研究,比如韩国整容业发达,是因为在韩国有一种社会认知:长得漂亮意味着拥有其他美好品质,打扮好、长得好的人可能更精明或有更多权力,在事业等方面表现更好,这就像晕轮效应,因外貌给人加上光环。这种社会认知与国家或集体的经济发展水平、社会形态、历史发展阶段都有关系。

在我国,这种现象在21世纪后变得比较夸张,担心容貌影响自身发展只是表象。从我的接触来看,有``容貌焦虑''的人,可能自尊水平很低,无法确定自己不管化不化妆都能被他人接受,所以必须呈现出完美的一面才放心。他们的安全感也可能较差,或许还曾因他人以审视的目光对待自己而受伤,比如社会推崇两米的身高为理想型,很多人达不到,这种社会潮流很容易形成,也很容易用一种社会认知去打击一群人。但每个人的承受能力不同,有的人由于成长经历,没有强大的内心,真的难以承受。


问:为了给人留下良好的第一印象,就刻意打扮自己,是否是一种``容貌焦虑呢''

答:从社会心理学的角度来看,第一印象的确很重要。比如面试,你没多少机会深入了解一个人,这时为抓住机会,展现出自己最好的一面(外貌只是其中一小部分,还有其他方面)就十分必要,而且要在短时间内做到。

不过,这里有个程度差异的问题。是仅仅为了这次机会才这么做,还是每次出门都必须如此呢?如果是后者,可能就有问题了。我听说有人因为没打扮好就不想出门,这会减少他们的人际接触。人有社交的天性,有向外拓展的自然倾向。

其实不只是容貌,很多因素都可能让我们不愿行动。但如果容貌问题严重到让你无法自由做自己,甚至不管什么机会都不想把握了,这就有些严重了。这样反而会失去很多接触人和资源的机会。

我们可以有意识地利用容貌,比如有人说你摘了眼镜会更漂亮,你可以尝试这样做。但要知道,容貌只是我们社会资本中很小的一部分。就像在你们理工科实验室,摆不摆花瓶有什么重大影响呢,研究成果才是关键。

``容貌焦虑''相当于在社会交往中过度放大了容貌的权重。因为在人的自我认知里,与外界相关且最直接的边界就是容貌,身体是最贴近自己的部分,这是无法逃避的。所以,用容貌贬低一个人很容易,当一个人焦虑到恨不得换张脸的程度时,这种焦虑也极易毁掉一个人。毕竟,身体是与外界接触的最前沿,是给别人留下第一印象的关键。


问:大学生朋友圈里,``美颜''照片泛滥,这会加剧还是缓解大学生``容貌焦虑''呢

答: 你要是喜欢刷朋友圈,就会发现朋友圈就像一个大秀场。大家都在展示自己生活中美好的一面,不仅是自己漂亮的照片,还有生活各个方面的精彩瞬间。

这让我想到之前提到的问题,关键在于每个人的内核是否稳定。对于那些对自己缺乏清晰认知、难以接纳自己的人来说,他们很容易受到外界标准的影响。比如,当有一种``两米才是最美''这样荒谬的标准出现时,他们一旦达不到,就会有被否定的感觉。但如果一个人内核稳定,对自己和世界有更深入的了解,那么即便看到大家都在精心打造朋友圈,也不会为之烦恼。

所以,这里面涉及到社会性信息能否激发个人情绪反应的问题,而其中有一个关键变量,就是个人内在的稳定性。这包括对自己的认知、对外界的理解以及自信心等多个因素。


问:如果把``容貌焦虑''比作一种传染病,您认为人们在社交过程中是怎么感染上这种``病''的呢?

答:你们想的这个问题其实很深刻。我们常常不自觉地陷入``容貌焦虑''的困境。我们很容易为自己塑造一个良好的形象,并且误以为别人是因为这个形象才和我们交往。当我们看到镜子中未经修饰的自己时,就会怀疑别人只是喜欢我们塑造的形象,而不是真实的自己。若是这样,那可不只是容貌焦虑了,还可能引发人际交往中的社交焦虑,甚至导致社恐和自我贬低。

确实,我们需要更具体地看待这个问题。现在我们太容易营造一个外在的``壳''了,大家也都通过这个``壳''相互接触。在社交和日常生活中,人们往往不会批判性地思考,只是跟着感觉走。在这个过程中,我们可能会发现,那些营造出精美``外壳''的人似乎得到了更多关注,而我们自己好像被忽视了。于是,我们就会想,是不是也应该加入这种营造``外壳''的潮流呢?这种现象是有影响力的,它可能会把很多人卷进去,我自己也受到了影响。比如,看到有人把某件衣服穿得很漂亮,我就以为自己穿上也会漂亮,买回来才发现不是那么回事。

所以,我们需要一种能力,就是区分别人为我们营造的虚幻泡影和真实的自己、真实的朋友。在这个时代,这种能力的提升变得愈发重要。如果这种能力不提升,我们就很容易被那些精心打造的``童话故事''所迷惑,跟着它们走。当我们在某个瞬间发现现实并非如``童话故事''那般美好时,如果我们的自我接纳程度很低,我们的心理防线就更容易被击溃。这是一个很不错的思考角度,值得深入研究。


问:如何克服``容貌焦虑''呢?

答:比较简单的方法就像我们刚才提到的,要塑造自己的内核。真正去发现自己的多个方面,要知道自己拥有很多资源。即便没有刻意打扮自己,我们依然可以对自己有良好的感觉。比如说,有一次同学们为我鼓掌,不是因为我化了妆,我甚至都不清楚自己做了什么,但同学们就是喜欢和我在一起。如果这样的经历越来越多,我们就越不需要依靠那些外在的堆砌了。
其实每个人都应该多花些心思在自己身上。现在,我们把太多精力放在关注别人创造的内容上了,比如短视频,却太少进行自我反思,将注意力回归自身。我们已经太久忽略了和自己内心的联系,甚至把这种自我关注看作是无用的,将其价值贬低、功利化。似乎什么都可以被功利化,但这样我们却失去了一些本真的东西。

当我们深入钻研这个话题到一定程度时,其实可以倡导一种新的社会文化。就像之前提到的那种单一的审美标准,比如``两米才是美''这种观念,我们可以推翻它,构建一个多元文化的世界,鼓励每个人更多地关注自身。如果我们去倡导这样的理念,社会会不会因此发生一些改变呢?会不会让那些深受容貌焦虑困扰的人得到解脱呢?我们可以试着把问题反过来思考。

就像我们之前在做女性相关宣传的时候,以一个知名女性内衣品牌为例,在它的广告里,不同高矮、胖瘦、民族、肤色的女性穿上内衣都展现出自信。如果我们多传播这样的信息,有没有可能给社会带来一些积极的改变呢?所以,除了个人要塑造稳定的内核,我们还需要创造不一样的文化,这样才能让更多人摆脱困扰,让大家不再为此担忧。


问:从客观角度来讲,``容貌焦虑''会受那些因素的影响呢?

答:刚才我提到,从我们平时的咨询案例、临床研究以及心理学研究来看,有很多关于容貌焦虑和其他心理学变量关系的论文。比如涉及胖瘦,还有心理学中身体的自我监控或自我认知这一概念(这里不只是容貌,是更学术化的概念)。

那这个身体的自我监控或自我认知和什么有关呢?和一个人童年的成长环境有关,比如是否在安全的依恋环境中成长。在安全依恋环境下,孩子能被理解、被关注,当他们去尝试新事物、冒险时会得到支持,拥有良好的关系支撑,这样他们会更自信,愿意在人际关系中探索,即便遭遇挫折也不会轻易否定自己。这样的人自尊更稳定。不过,我不想给大家塑造一种新的童话感,毕竟世界上没有完美的人,我们在不同情境下状态各异,有时自我感觉不太好。但在心理学测量中,会有一个相对稳定、有信度和效度的值。安全依恋型的人对自我的看法相对更积极肯定,受文化影响也会少一些。

我还觉得文化在这其中非常重要。我看过的研究显示,女性的身体自我监控和其所处环境相关。比如在重男轻女文化环境中成长的女性,从出生起就被认为没有价值,她只能想办法让自己成为有价值的``商品'',其中一种途径就是让自己变得更漂亮、更有吸引力,这是多么可悲的思路。所以,文化中男女平等、多元发展等因素是有影响的。


问:追星是否会影响一个人``容貌焦虑''的程度?

答:你们都刚从青春期过来,应该知道那是一个自我认知极不稳定的阶段。在这个时期,我们很容易追星,因为父母已经不太可能成为我们的偶像了,同学也不一定能成为我们效仿的对象。这时候,社会上那些被塑造出来的形象,就很容易成为我们衡量自己或者塑造自己的标准。要是这些形象不能给我们提供多元的视角,我们就可能受到负面影响。

其实我们刚才讨论的就是 ``造星''``造神'' 现象,像这些明星,他们都像是被塑造出来的 ``神'',只是明星的影响力是面向全社会的。不过现在有些明星做得很好,他们能分得很清楚,比如在舞台上的时候,会把自己打扮得光鲜亮丽,但在生活中,像有些女明星被偷拍到时,模样非常朴素。这种行为有着积极的意义,符合我们刚才提到的多元文化内涵。


结论:``容貌焦虑'',并非专业医学诊断之名,却似一场无孔不入的``心理风暴'',肆虐于生活各个角落。从本质而言,它源起于那虚幻缥缈、无迹可寻却又被大众狂热追捧的``美貌标准''。
谈及日常情境,刻意打扮求第一印象本无可厚非,面试等场合需展现最佳态,但若陷入``不精致不出门''极端,便是过度放大容貌权重,致社交畏缩、错失良机,毕竟容貌只是社交资本``冰山一角'',却常被当作评判``利刃'',伤人于无形。大学生朋友圈``美颜秀场'',加剧或缓解焦虑全看个人``内核''稳不稳,内心笃定者淡然处之,迷茫者则易被``两米最美''这类荒谬标准裹挟,陷入自我否定泥沼。

如何突围?一方面,个体要重塑强大``内核'',挖掘自身多元价值,珍视那些无关容貌被认可时刻,把目光从外界纷扰拉回内心本真;另一方面,社会亟需摒弃单一审美``毒瘤'',构建多元包容文化,像倡导女性内衣广告展现多元身材自信风采那般,打破刻板禁锢。追根溯源,童年成长环境影响身体自我监控与认知,安全依恋助自尊稳固,免受文化``侵蚀''。

\subsection{采访同学}
问:你认为你存在容貌焦虑吗?你是如何判断的?

答:存在。我觉得这是基于一定的水准,内心有着想往更加向善向好的容貌方向发展的渴望,同时也夹杂着一种焦虑感和攀比心理吧。

问:请你描述一下你心目中最完美的男性外貌。

答:我觉得应该是既有干净清爽、五官端正整洁的脸庞,又具备强壮的肌肉和高挑的身材。

问:你认为你的这种审美认识是如何产生的呢?

答:可能是源于平时看的一些篮球赛事片段,还有对电影明星形象的印象吧。

问:你会花费多少时间精力在整理容貌上呢?

答:一般就是早晨起来的时候打理一下,洗完澡也会打理打理,平常要是碰到镜子,也会不自觉地照一照。

问:你觉得这会对你的正常生活造成影响,或者让你产生困扰吗?

答:我认为基本不会。

问:你认为容貌焦虑在你身上有什么主要的表现呢?

答:主要就是经常会去关注自己的发型,在意脸上是不是清爽干净这类事儿。

问:你认为容貌焦虑对你的恋爱产生了何种影响呢?

答:我觉得从积极方面来讲,可能会让我的另一半看着更赏心悦目,让她产生一种良好的感觉,对我自己而言,也会有一种心理上的安稳感,总体算是正面影响吧。不过,要是从生活角度说,有时候因为过分关注容貌,像发型乱了的时候,也会有点头疼。

问:你觉得这对你的女朋友或者身边的其他人产生了什么影响吗?

答:我觉得对我女朋友来说应该是好事,而对于身边其他人,保持良好的容貌,算是对他人的一种尊重吧。

问:这对你女朋友她的心理有没有产生一些变化呢?

答:我觉得容貌这块,主要是我自己比较在意,对她来说可能没那么重要。

问:为什么似乎你比起绝对的颜值更加关注自己在人群中的百分比呢?

答:嗯,确实是这样。怎么说呢,就跟比成绩似的,自己达到一个高度,那只是自身的绝对水平,只有跟别人比,不断进步,才能脱颖而出啊,脱颖而出不就是要比别人更厉害嘛。

问:啊,嗯,你有想过改变你容貌焦虑的现状吗?

答:我目前还没有这个想法。

问:你认为容貌焦虑这件事本身对你有没有什么正面的影响?

答:正面影响肯定有啊,因为有想变得更好的念头,就会促使自己往好的方向发展。比如说,我会经常去跑步、健身,通过锻炼来提升自己的气质。

问:最后请你对容貌焦虑做一个尽可能全面的评价。

答:在我看来,容貌焦虑就是基于自身现有的一个水准,怀揣着向更高层次攀升的想法与追求。它具有两面性,有时候确实会带来一些负面情绪,但更多时候,对我而言它是促使我积极向上的动力,让我更愿意去做那些健康阳光的事儿。所以,在日常生活里,我们得妥善处理好这其中的利弊关系,争取把积极影响最大化,避免产生负面影响,这样才能拥抱更好的自己。

好的感谢你的参与。

总结:容貌焦虑,宛如一把双刃剑,深深嵌入该同学的生活与心境之中!
他坦然承认自身存在这一焦虑,怀揣着向更优容貌奔赴的热望,其间杂糅着焦虑感与攀比心。其理想中的男性模样,是面庞干净清爽、五官规整,兼具强壮肌肉与高挑身形,这般审美源自篮球赛场的热血片段和电影明星的璀璨形象。日常里,晨起、浴后的打理,逢镜自照,已成习惯,所幸并未对生活造成过多纷扰。在恋爱场域,容貌焦虑恰似调味剂,让伴侣赏心悦目、予己心安,可偶尔发型凌乱时,也会添一丝烦恼。对女友及旁人而言,良好容貌是尊重的外化,女友虽未将其容貌过分挂怀,可受访者仍执着于在人群里的``容貌排位'',恰似追逐学业成绩般,力求脱颖而出。当下,他暂无改变这一现状的念头,因容貌焦虑亦有正向驱动,鞭策他跑步健身、雕琢气质。归根结底,容貌焦虑有其双面性,会滋生负面情绪,却更多时候是积极奋进的引擎,促使他投身健康阳光之举,我们理当巧妙权衡利弊,让其助力我们奔赴更卓越、更坦然的自己,而非沦为其``囚徒''!