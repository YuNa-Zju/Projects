考虑到本次研究性学习与大学生有着密切联系,所以我们采用了问卷调查的研究方法,并在选取典型样本进行更加细致的采访调查,旨在更好的了解``容貌焦虑''在大学生中的影响情况。
我们还采访了专业的老师,希望从中了解更加专业成熟的想法。
\subsection{查找资料}
在网络上查找相关资料,论文,了解``容貌焦虑''问题的相关信息
\subsection{问卷调查}
通过发放问卷并统计数据,获得关于大学生群体容貌焦虑问题的相关情况
\subsection{采访相关人士}
采访有典型烦恼的同学,以及有经验的老师,来获取解决的方法
\subsection{数据汇总研究}
分析网络上收集的数据,收集的数据,得出结论
\subsection{研究计划}
\begin{table}[H]
    \caption{研究计划时间表}
    \begin{tabularx}{\textwidth}{>{\raggedright\arraybackslash}X
        | >{\raggedright\arraybackslash}X}
    \hline
    \textbf{时间}                                                                             & \textbf{分工}                                                                                                         \\ \hline
    \begin{tabular}[c]{@{}l@{}}\textbf{秋学期第4周}\\ 查找资料\end{tabular}                      & 查找``容貌焦虑''的说法从何而来                                                                                                                    \\ \hline
    \begin{tabular}[c]{@{}l@{}}\textbf{秋学期第5周}\\ 制作问卷\end{tabular}                        & \begin{tabular}[c]{@{}l@{}}调查自认为有``容貌焦虑''和不认为自己有``容貌焦虑'' \\ 的主要群体(以年龄段、审美类型、家庭氛围等进行分类)\\ 调查诱发``容貌焦虑''的社会环境因素(广告、媒体等) \\ 和个人心理因素\end{tabular} \\ \hline
    \begin{tabular}[c]{@{}l@{}}\textbf{秋学期第6、7、8周}\\ 进行采访\end{tabular}                    & \begin{tabular}[c]{@{}l@{}}调查``容貌焦虑''的具体表现和影响 \\ 调查``容貌焦虑''主要集中在总结出来的群体的原因 \\ 调查大学生群体对``容貌焦虑''的认识\end{tabular}                       \\ \hline
    \begin{tabular}[c]{@{}l@{}}\textbf{冬学期第1、2、3、4周}\\ 对前期收集到的数据、信息、资料等归纳分析\end{tabular} & 整理解决``容貌焦虑心理''的具体方法                                                                                                          \\ \hline
    \end{tabularx}
\end{table}