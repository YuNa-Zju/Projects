\subsection{容貌焦虑在大学生群体中的情况}
\begin{enumerate}[leftmargin=7em]
    \item \textbf{过度关注他人评价}

    大学生容貌焦虑群体大多受到过来自同龄人和社交圈中他人消极评价的影响,尤其是在社交平台上,对于他人对自己外貌的评价、点赞数、评论等过度敏感,易产生焦虑和自卑感。

    \item \textbf{社交回避或焦虑}

    由于对自己容貌的不自信,大学生容貌焦虑群体可能会出现社交回避的情况,避免在人群中出镜或参与集体活动,也可能在社交场合中产生紧张、不自然的情绪。

    \item \textbf{心理健康问题}

    容貌焦虑的长期积累可能导致一些心理健康问题的出现,如焦虑症、抑郁症等。

    \item \textbf{容貌的``社会对比''和``光环效应''}

    大学生容貌焦虑群体大多会将自己与他人反复进行外貌上的对比,从而产生自我怀疑。此外,他们大多会将容貌与其他优秀品质进行联系,认为外貌好的人在社交和事业中可能更有优势,进而产生焦虑。

\end{enumerate}

\subsection{容貌焦虑在大学生群体中的成因}
\begin{enumerate}[leftmargin=7em]
    \item \textbf{个人心理因素}
        \begin{enumerate}
            \item 未形成较为强大的心理内核;
            \item 未形成健康而成熟的价值观、恋爱观、就业观、世界观和审美。
        \end{enumerate}
    \item \textbf{社会环境因素}
        \begin{enumerate}
            \item 夸张、悬浮的影视作品的发行;
            \item 不良商家和网红为盈利发布的宣传``畸形审美''的广告、视频\autocite{__2022-4};
            \item 社会价值评价体系的不全面和对``成功''的片面定义。
        \end{enumerate}
\end{enumerate}

\subsection{解决途径}
\begin{enumerate}[leftmargin=7em]
    \item \textbf{个人层面}
    \begin{enumerate}
        \item \textbf{增强自我认知与自我接纳}

        个体可通过深入的自我探索,将注意力从容貌转移到自我其他方面的价值。例如,通过多多参加各类活动探索自己的兴趣、技能和内在品质,认清楚容貌只是人类众多特质之一,且其相对价值不应被过度放大。通过这种方式,个体可以更好地接纳自身的外貌,并树立自信心。

        \item \textbf{积极阅读、吸收有关健康审美的相关专业人士的书籍、观点。}

        \item \textbf{积极寻求心理咨询的帮助}

        不一定有很严重心理问题时采取进行心理咨询,心理咨询也是帮助个体形成健康认知、缓解消极情绪的重要渠道。

        \item \textbf{多接触真实的社会而非仅停留在文艺作品中}

        大学生群体可以通过实习工作接触真实工作环境,形成对社会竞争力的正确认知;可以尝试鼓起勇气和有好感的人进行一定的交流,在经过深入认识后再形成恋爱关系,同时也在和真实的人的接触中树立健康的恋爱观。
    \end{enumerate}
    \item \textbf{社会层面}
    \begin{enumerate}
        \item \textbf{推广健康、积极的审美观念}

        社会应倡导多元化、健康的审美标准,摒弃狭隘的``标准化''美学,尊重每个人的独特性。通过对媒体和广告的正向引导、纠正,推广更加宽容与包容的美学标准,从而鼓励个体从多角度理解和欣赏美的多样性。

        \item \textbf{构建多元价值体系}

        在社会文化中,应积极传播和构建更加多元的价值体系。人类的价值不仅仅体现在外貌、成绩或社会地位上。每个人都有其独特的内在魅力和贡献,社会应当关注个体的人格、道德品质、社会责任等更深层次的内在价值。这种方式可以帮助个体建立更加健康的价值观,减少对容貌的过度关注。

        \item \textbf{倡导``人性回归''}

        新时代青年与有识之士应共同倡导社会关注回归人的本质,强调人与人之间的真实、深刻的情感连接与理解,而非仅仅基于外貌等浮于表面的表现的评判。社会也应为个体提供更多能展现自己内在的真善美的机会,使个体不仅仅在外貌、成绩等表面特征上寻求认同。

        \item \textbf{社会与教育层面的支持}

        学校、社会组织等应加强心理健康和自我探索教育和引导,帮助个体更好地理解容貌焦虑的成因与影响,提供心理疏导和支持,帮助学生树立更加全面、均衡的自我认知,理解人的多维价值,从而减少容貌焦虑现象的发生。
    \end{enumerate}
\end{enumerate}
